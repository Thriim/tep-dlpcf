\documentclass{beamer}
 
\usepackage[utf8]{inputenc} 
\usepackage[T1]{fontenc}
\usepackage{lmodern}
\usepackage{graphicx}
\usepackage[french]{babel}

\usetheme{Warsaw}

\begin{document}

\begin{frame}[fragile]
\frametitle{Classical Type systems don't assure termination}


\begin{verbatim}
add = fix f y z => ifz y then 0 else (succ (f (pred y) z)
\end{verbatim}

\begin{verbatim}
divergef = fix f y z => ifz y then 0 else (succ (f y z))
\end{verbatim}
Both have type Nat -> Nat but one diverges and the other doesn't
\end{frame}

\begin{frame}[fragile]
\frametitle{Classical type systems don't inform about complexity}
Which one is best? 
\begin{verbatim}
 identity = x => x 
\end{verbatim}
\begin{verbatim}
madIdentity = fix f x => 
                ifz x then 0 else (succ (f (pred x))
\end{verbatim}
\end{frame}

\begin{frame}[fragile]
\frametitle{Two improvements}
\begin{itemize}
\item Dependent types
\begin{figure}
\begin{verbatim}
 add : Nat[a] => Nat[b] => Nat[a+b]
\end{verbatim}
\end{figure}
\item Linear types \\
\begin{figure}
\begin{center}
$\sigma_{n}~\multimap~\tau$
\end{center}
\end{figure}
\end{itemize}
\end{frame}

\begin{frame}[fragile]
\frametitle{Putting them together}
\begin{itemize}
\item Dependent types provide information about the function's inputs and outputs

\item Linear types provide information about how many times a parameter is
  copied or a function is used (substitution)

\item Taking the information from both + giving an extra context
($\varepsilon$) allows to infer a function's complexity and thus termination
\end{itemize}
\end{frame}

\begin{frame}[fragile]
\frametitle{A simple example}
\begin{verbatim}
  add: fix f y z => ifz y then 0 else (succ (f (pred y) z)
\end{verbatim}
\begin{itemize}
\item Linear dependent type 

  $Nat[a]~[c < 1]~\multimap~Nat[b]~[d < 1]~\multimap~Nat[a + b]$

\item  How many substitutions will there be? 

  Informally: substitutions = subst-of-recursive-function +
  (subst-per-recursive-call * number-recursive-calls)

\end{itemize}
\end{frame}

\begin{frame}[fragile]
\frametitle{A simple example (2)}

\begin{verbatim}
  add 3 4 = succ(add 2 4)
  -> add 2 4 = succ(add 1 2)
  -> -> add 1 4 = succ(add 0 4)
  -> -> -> add 0 4 = 0
\end{verbatim}
\begin{itemize}
\item I (how many recursive calls) = If b < f then 1 else 0
\item subst-of-recursive-functon = <>0..f for b of I
  % Epsilon will save us:  
  %add is substituted 4 times (last time it's substituted even if its not
  %called)
\end{itemize}
\end{frame}

\begin{frame}
\frametitle{Compilation in an abstract machine}

d$l$PCF can be compiled in a CEK abstract machine.

\medskip

CEK composition:
\begin{itemize}
\item A stack $\pi$.
\item A closure $c~=~<t;~\varepsilon >$, with $t$ a term and $\varepsilon$ the
  environment, to interact with the stack.
\end{itemize}

\end{frame}

\end{document}

% Slide 3
Two improvements over classical type systems


The type carries more meaningful information about the transformations
perform inside a function.

e.g

sized arrays => ArrayList<Integer, Size>  

- Linear types

Come from linear logic, allow to  define variables that cannot be shared
(e.g iterator) or that can be shared only "n" times.

e. g

Types define the number of times a variable is "used"

% Slide 4
Putting them together

- Dependent types provide information about the function inputs and outputs

- Linear types provide information about how many times a parameter is
  copied or a function is used (substitution)

- Taking the information from both + giving an extra context (Epsilon) allows to
infer a function's complexity and thus termination


A simple example

  add: fix f y z => ifz y then 0 else (succ (f (pred y) z)

  dependent type: (Nat[a] => (Nat[b] => Nat[a + b]))

  How many substitutions will there be? 

  Informally: substitutions = subst-of-recursive-function +
  (subst-per-recursive-call * number-recursive-calls)

% Slide 5
A simple example (2)

  Epsilon will save us:  

  I (how many recursive calls) = If b < f then 1 else 0

  subst-of-recursive-functon = <>0..f for b of I

  e.g add f=3 g=4

  add 3 4 = succ(add 2 4)
  -> add 2 4 = succ(add 1 2)
  -> -> add 1 4 = succ(add 0 4)
  -> -> -> add 0 4 = 0

  add is substituted 4 times (last time it's substituted even if its not
  called)

A simple example (3)

  number-recursive-calls = subst-of-recursive-function
  
  e.g add =>

  subst-per-recursive-call = 2 = 1 (call to succ) + 1 (call to pred)


  So the final cost is 
  
  cost = 4 + (2 * 4) = (f + 1) + 2 * (f + 1) 

%Slide 6
An abstract machine for PCF

CEK : this abstract machine is really simple and specialized for lambda calculus

There are 2 things to consider : the stack and the term to evaluate, with its
actual environment. 

It's kind of simple : for example, when evaluating an application, the closure
is first push on the stack, and then, since only a value remains, the stakc is
poped and the result is applied to the remaining value.

CEK is used in particular with call-by-value lambda calculus
