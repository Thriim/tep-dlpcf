\documentclass[a4paper,12pt]{report}

\usepackage[utf8]{inputenc}
\usepackage[francais]{babel}
\usepackage{verbatim}

\title{Linear Dependant Types in a Call-by-Value Scenario \\
- \\
Personal notes}

\author{Pierrick COUDERC}

\begin{document}

\maketitle

\chapter{Notes}

\section{Introduction}

\section{Linear Dependent Types, Intuitively}

Lambda-calculus in PCF Type System: 
\begin{itemize}
\item fix: fixpoint operator, for recursive functions
\item ifz: if zero
\item s: succ
\item p: pred
\end{itemize}

Functions presented:
\begin{verbatim}
let rec dbl x =
  if x = 0 then x
  else succ @@ succ @@ dbl @@ pred x

let rec div x = (* diverges *)
  if x = 0 then x
  else succ @@ succ @@ div x
\end{verbatim}


\chapter{Article}

\paragraph{Abstract}

\section{Introduction}

Types and types systems are extremely useful and powerful to garanty some
prooperties on programs, principally that a program ``cannot go wrong''
(i.e. there shouldn't be types incompatibilities and crash during the
execution), especially with languages using strong type systems. However, those
types systems aren't able to catch some properties, like the termination of a
program and the time needed to be executed, i.e. the computational
complexity. For exemple, we can consider this simple factorial function:

\begin{verbatim}
let rec fact n = if n = 0 then 1 else n * fact (n-1)
\end{verbatim}

We can easily prove that such a recursive function will terminate if we consider
only the natural integers, but there is no other way to check it before
executing the function.

The idea of the linear dependent types is to extend a type system to add the
ability for the typer to check if there is no diverging, an approximation of the
computational complexity and even a behavior of the function. Plotkin's PCF,
which is a lambda -calculus extended with natural integers, will be used as a
language and types (only Nat and functions on Nat actually), adding the two
following properties :
\begin{itemize}
\item \textbf{Linearity} : being able to catch the number of evaluation required
  for a term t, this way giving an idea of the computational complexity the
  program.
\item \textbf{Dependency} : the possibility to distinct copies of a term with
  distinct types.
\end{itemize}

Such PCF programs, typed with dlPCF, can then be interpreted by Krivine's
Abstract Machine for a Call-by-Name evaluation strategy, but this setting is not
really efficient, and few languages are actually using this strategy. the idea
is then to extend the original dlPCF (called $ dlPCF_{N} $) with the ability to type
programs with a call-by-value strategy, namely $ dlPCF_{v} $, and use the Felleisen and
Freidman's CEK machine to interpret it.

\section{Linear Dependent Types}

First of all, let's consider a simple program written in PCF (whose syntax is
recalled in \ref{pcf-syntax} :

\begin{center} 
  dbl = fix~f.$\lambda $x.~ifz~x~then~x~else~s(s(f(p(x)))) 
\end{center}

Simply, this function will return the double of its argument. In the PCF type
system, it will be typed as Nat -> Nat. However, there are no other
information about its behavior and its meaning in this type, since, for example,
a lot of unary functions can be typed the same way (an identity function for
example, with a little trick to be sure it will be typed with Nat). In
particular, if we change p(x) by x, this function will have the same type, but
it will diverge. We may need a type system that will be able to catch those
diverging behavior, if that's ever possible.

First of all, we can extend the types with informations on values, like for
\emph{dbl} : $\tau$ = Nat[a] -> Nat[2 x a]. This way, we can see that the type
contains the exact behavior of the function, however it will be rarely to
specify such a precise type. it is necessary then to add a \emph{imprecise
  type}, like for example \textbf{Nat[2, 4]} which is a natural integer whose
value is between 2 and 4.

\medskip

Secondly, a linear dependent type system has an interesting proprety, which
captures the termination behaviour for functions that are fully recursive. An
good example was the one we provided before, but in a classic type system, their
behaviour cannot be differentiated. However, one terminates, while the other
diverges. This is a property that sized types and dependent types satisfies. To
do so, the type system, by trying to prove termination, could also gives the
time and space consumption. Using the ideas from linear logic, we could, for
example give type the following way :

\begin{center}
$\vdash_{I}$ dbl : Nat $\rightarrow$ Nat
\end{center}

In this type, I is the cost to compute dbl, and, since it can be given, returns
the fact that the function doesn't diverge.

The idea behind dlPCF merges the imprecise types and the computational costs,
combined with ideas from the bounded linear logic. This one allows a practical
way to represent the number of times a function uses its argument, with the
following syntax as instance :

\begin{center}
!$_{n}\sigma$
\end{center} 

\section{dlPCF}

\section{Krivine's Abstract Machine and CEK}

\end{document}

% Local Variables:
% compile-command: "rubber -d notes.tex"
% ispell-local-dictionary: "english"
% End:
